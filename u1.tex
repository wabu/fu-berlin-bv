\include{headerueb}
\include{header}
\include{info}


\newcommand{\nr}{1}

\begin{document}
\section*{Aufgaben 1 - Hadamard}
Wie in der Vorlesung beschrieben, wird die Hadamard-Matrix rekursiv generiert:

\lstset{language=matlab}
\begin{lstlisting}[]
function matrix = hadamard(dim)
    if dim <= 1
        matrix = [1];
    else
        if mod(dim,2) != 0
            printf ('unsupported dimension\n');
        else
            m=hadamard(dim/2);
            matrix=[m,m;m,-m];
        end
    end
end
\end{lstlisting}

Die Transformation und R\"ucktransformation lassen sich dann in wenigen Schritten realisieren:


\lstset{language=matlab}
\begin{lstlisting}[]
H = hadamard(rows);
# normieren
H = H/sqrt(rows);

I_trans = H*double(I_in)*H;
I_trans = reduce(I_trans,0.2);
I_out = H*I_trans*H;
\end{lstlisting}

\section*{Aufgabe 2 - Fourier} 

\section*{Aufgabe 3 - Verlust}
\paragraph{Hadamard}
Es ist m\"oglich den \"uberwiegenden Teil der Elemente in der transformierten 
Matrix zu entfernen, ohne sichtbare Unterschiede zu bemerken. Ab einem gewissen
Punkt jedoch, setzt Rauschen ein, der sich zunehmend verst\"arkt. Die Differenz
zwischen Original und Rekonstruktion zeigt, das Verluste prim\"ar an Frequenz\"uberg\"angen
entstehen.

\paragraph{Fourier}

\end{document}
